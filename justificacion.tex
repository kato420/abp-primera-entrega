\section{Justificaci\'on}
\label{sec:justificacion}

El crecimiento acelerado de la demanda de veh\'iculos el\'ectricos ha impulsado la necesidad de sistemas de carga m\'as eficientes y estables. Uno de los principales problemas en estos sistemas es la generaci\'on de arm\'onicos, que deterioran la calidad de la energ\'ia el\'ectrica y afectan negativamente tanto el funcionamiento del cargador como la durabilidad de la bater\'ia. En este contexto, el presente estudio busca aplicar la funci\'on de transferencia como una herramienta de an\'alisis y optimizaci\'on de filtros electr\'onicos, con el fin de mitigar dichos arm\'onicos y mejorar el rendimiento global del proceso de carga.

Adem\'as, esta investigaci\'on contribuye al avance tecnol\'ogico del sector del transporte sostenible, proporcionando una base te\'orica y pr\'actica para el dise\~no de sistemas de carga m\'as seguros y confiables. El enfoque basado en la funci\'on de transferencia permite un an\'alisis m\'as preciso del comportamiento din\'amico de los filtros, lo cual resulta esencial para la mejora continua de los cargadores de bater\'ias y, por ende, de la infraestructura vehicular el\'ectrica.
    
\newpage
