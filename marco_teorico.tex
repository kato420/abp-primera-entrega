\section{Marco te\'orico}
\label{sec:marco_teorico}

En el contexto de la movilidad el\'ectrica, el estudio de los sistemas el\'ectricos ha cobrado relevancia debido al creciente uso de veh\'iculos el\'ectricos, cuyos cargadores est\'an expuestos a perturbaciones como los arm\'onicos. Estas fluctuaciones en la corriente el\'ectrica afectan la calidad de la carga, reduciendo la eficiencia del sistema y acortando la vida \"util de componentes como las bater\'ias. Para mitigar estos efectos, se emplean herramientas matem\'aticas como la funci\'on de transferencia y la Transformada de Laplace, que permiten modelar el comportamiento del sistema y dise\~nar filtros electr\'onicos adecuados (Zhang, Wang, \& Li, 2021). Este marco te\'orico aborda los principios esenciales para dicho an\'alisis, incluyendo las leyes de Kirchhoff, fundamentales en el estudio de circuitos el\'ectricos.

\subsection{1. Conceptos}

\subsubsection{1.1 Funci\'on de Transferencia:}
La funci\'on de transferencia es una herramienta matem\'atica utilizada para describir la relaci\'on entre la entrada y la salida de un sistema din\'amico en el dominio de Laplace. Es especialmente \"util en la ingenier\'ia de control y en el an\'alisis de sistemas el\'ectricos, donde permite modelar la respuesta del sistema ante perturbaciones como los arm\'onicos. En el caso de los cargadores de bater\'ias de veh\'iculos el\'ectricos, la funci\'on de transferencia ayuda a modelar el comportamiento de los filtros electr\'onicos que mitigan los arm\'onicos presentes en el sistema de carga (Bola\~nos, s/f).

\subsubsection{1.2 Transformada de Laplace:}
La Transformada de Laplace es una herramienta esencial en el an\'alisis de sistemas din\'amicos, permitiendo convertir ecuaciones diferenciales en ecuaciones algebraicas. Este enfoque simplifica la resoluci\'on de problemas complejos en circuitos el\'ectricos y otros sistemas din\'amicos. En el contexto de los cargadores de bater\'ias, la transformada de Laplace es \"util para analizar c\'omo los filtros afectan la se\~nal de entrada, permitiendo el dise\~no de soluciones m\'as efectivas para mitigar los arm\'onicos (L\'azaro et al., 2016).

\subsubsection{1.3 Leyes de Kirchhoff:}
Las leyes de Kirchhoff son esenciales para la evaluaci\'on de circuitos el\'ectricos. As\'i mismo, la ley de corrientes de Kirchhoff (LCK) establece que la suma de las corrientes en un nodo es cero, mientras que la ley de voltajes de Kirchhoff dicta que la suma de los voltajes en una malla cerrada es cero. Estas leyes son clave para entender el flujo de energ\'ia en los sistemas de carga de bater\'ias y son esenciales en el dise\~no de filtros que ayudan a reducir los arm\'onicos (Arias, 2015).

\subsubsection{1.4 Arm\'onicos en Circuitos El\'ectricos:}
Los arm\'onicos son componentes de frecuencia m\'ultiplo de una se\~nal fundamental que se superponen a la se\~nal de corriente o voltaje en un sistema el\'ectrico. Estas frecuencias adicionales son generadas por la no linealidad de los dispositivos conectados al sistema, como convertidores de potencia o cargadores de bater\'ias, que generan distorsiones en la forma de onda de la corriente o el voltaje. Estos arm\'onicos afectan negativamente la calidad de la se\~nal el\'ectrica y pueden interferir con el funcionamiento adecuado de los componentes del sistema (Abundis, 2016).

\subsection{2. Campo de Aplicaci\'on Espec\'ifico}
El campo de aplicaci\'on espec\'ifico de este proyecto es el sistema el\'ectrico de carga de bater\'ias para veh\'iculos el\'ectricos. Este sistema es un ejemplo de un sistema din\'amico que se ve afectado por la presencia de arm\'onicos, los cuales son fluctuaciones en la corriente que pueden afectar negativamente la eficiencia de la carga y la vida \"util de las bater\'ias. El estudio se enfoca en c\'omo los filtros electr\'onicos, modelados a trav\'es de ecuaciones diferenciales y utilizando la funci\'on de transferencia, pueden mitigar estos arm\'onicos, mejorando as\'i el rendimiento del sistema de carga.

\subsubsection{2.1 M\'etodos de Resoluci\'on:}
\textbf{I. An\'alisis mediante la funci\'on de transferencia:}\\
Para resolver el problema de los arm\'onicos en los sistemas de carga, se utilizar\'a la funci\'on de transferencia para modelar el comportamiento de los filtros electr\'onicos. Esto permitir\'a predecir la respuesta del sistema ante perturbaciones y optimizar el dise\~no de los filtros.
\vspace{1em}

\textbf{II. Resoluci\'on mediante la Transformada de Laplace:}\\
Se aplicar\'a la Transformada de Laplace a las ecuaciones diferenciales que describen los circuitos el\'ectricos, transform\'andolas en ecuaciones algebraicas m\'as f\'aciles de manejar (Fern\'andez et al., 2022). Este m\'etodo es crucial para analizar c\'omo los filtros electr\'onicos pueden mitigar los arm\'onicos y mejorar la eficiencia del sistema de carga.
\vspace{1em}

\textbf{III. Simulaci\'on y an\'alisis de estabilidad:}\\
Se realizar\'an simulaciones del sistema de carga de bater\'ias, modelado mediante la funci\'on de transferencia y las ecuaciones diferenciales, para evaluar el impacto de los arm\'onicos y la efectividad de los filtros. Adem\'as, se analizar\'a la estabilidad del sistema utilizando herramientas matem\'aticas basadas en las leyes de Kirchhoff y otros m\'etodos de resoluci\'on anal\'itica.
\vspace{1em}

\textbf{IV. Aplicaciones complementarias:}\\
Se revisar\'an estudios previos sobre estrategias de compensaci\'on activa y control adaptativo que permitan un mejor rendimiento en condiciones variables. Se tomar\'an como referencia modelos similares en microredes y sistemas HVDC (Rogers Acevedo, 2008; Universidad de Chile, 2015).

\newpage
