\documentclass[12pt]{article}
\usepackage[utf8]{inputenc}
\usepackage[spanish]{babel}
\usepackage{graphicx}
\usepackage{array}
\usepackage{geometry}
\geometry{margin=2.5cm}
\usepackage{titlesec}
\usepackage{xurl}
\usepackage{float}
\usepackage{fancyhdr}
\usepackage{url}        % <- Opcional pero útil para enlaces
\usepackage{hyperref}   % <- AQUÍ VA
\hypersetup{
  colorlinks=true,
  linkcolor=black,
  urlcolor=blue,
  breaklinks=true
}



\pagestyle{fancy}
\fancyhf{}
\rfoot{\thepage}

\begin{document}

% Portada sin numeración
\pagenumbering{gobble}
\begin{titlepage}
    \centering
    
    % Logo
    \includegraphics[width=0.3\textwidth]{utec.png}
    
    \vspace{0.3cm}
    \textbf{UTEC}\\
    Universidad de Ingenier\'ia y Tecnolog\'ia
    
    \vspace{1.5cm}
    
    {\Large\textbf{ABP - Primera Entrega}\par}
    
    \vspace{1cm}
    
    {\bfseries T\'itulo:}\\
    \begin{minipage}{0.9\textwidth}
    \centering
    \textit{\"Optimizaci\'on de filtros el\'ectricos en cargadores de bater\'ias para veh\'iculos el\'ectricos mediante la aplicaci\'on de funciones de transferencia.\"}
    \end{minipage}
    
    \vspace{1cm}
    
    {\bfseries Curso:} Ecuaciones Diferenciales Ordinarias (CC2101)\\
    {\bfseries Teor\'ia:} 1.01 - Grupo 6\\
    {\bfseries Profesora:} Patricia Reynoso Quispe
    
    \vspace{1.5cm}
    
    \renewcommand{\arraystretch}{1.3}
    \begin{center}
    \begin{tabular}{|>{\raggedright\arraybackslash}p{6cm}|c|c|}
    \hline
    \textbf{Integrantes} & \textbf{C\'odigo} & \textbf{Participaci\'on \%} \\
    \hline
    Choque Shuan, Katherine Massiel & 202410623 & 100 \\
    Huaman Yay, Alexis & 202210507 & 100 \\
    Oceda Chavez, Elizabeth Emperatriz & 202220111 & 100 \\
    V\'asquez Bustamante, Mar\'ia Fernanda & 202220385 & 100 \\
    \hline
    \end{tabular}
    \end{center}
    
    \end{titlepage}

% Índice con numeración normal desde 1
\clearpage
\pagenumbering{arabic}
\setcounter{page}{1}
\section*{\centering \textbf{Índice}}

\begin{itemize}
  \item \textbf{Introducción} ..................................................... \pageref{sec:introduccion}
  \begin{itemize}
    \item Pregunta de investigación
  \end{itemize}

  \item \textbf{Objetivos} ......................................................... \pageref{sec:objetivos}
  \begin{itemize}
    \item Objetivo general
    \item Objetivos específicos
  \end{itemize}

  \item \textbf{Justificación} ..................................................... \pageref{sec:justificacion}

  \item \textbf{Marco teórico} .................................................... \pageref{sec:marco_teorico}
  \begin{itemize}
    \item 1. Conceptos
    \begin{itemize}
      \item 1.1 Función de Transferencia
      \item 1.2 Transformada de Laplace
      \item 1.3 Leyes de Kirchhoff
      \item 1.4 Armónicos en Circuitos Eléctricos
    \end{itemize}
    \item 2. Campo de Aplicación Específico
    \begin{itemize}
      \item 2.1 Métodos de Resolución
      \begin{itemize}
        \item I. Análisis mediante la función de transferencia
        \item II. Resolución mediante la Transformada de Laplace
        \item III. Simulación y análisis de estabilidad
      \end{itemize}
    \end{itemize}
  \end{itemize}

  \item \textbf{Formulación del problema} ......................................... \pageref{sec:formulacion_problema}

  \item \textbf{Referencias bibliográficas} ........................................ \pageref{sec:referencias}
\end{itemize}

\newpage


% Contenido del documento
\section{Introducci\'on}
\label{sec:introduccion}

El transporte es un componente esencial de la sociedad moderna, ya que facilita el movimiento de personas y bienes, impulsa la econom\'ia y favorece la interacci\'on social. Con el tiempo, los veh\'iculos se han convertido en el principal medio para satisfacer estas necesidades. Actualmente, el sector atraviesa una transformaci\'on impulsada por el auge de los veh\'iculos el\'ectricos, que, al funcionar con energ\'ia el\'ectrica en lugar de combustibles f\'osiles, representan una alternativa m\'as sostenible y contribuyen a la disminuci\'on de la contaminaci\'on ambiental. Seg\'un la Agencia Internacional de Energ\'ia (2023), la adopci\'on de veh\'iculos el\'ectricos podr\'ia reducir las emisiones de carbono en un 30\% para el 2040, lo que destaca a\'un m\'as la importancia de esta transici\'on.
\setlength{\parskip}{1em}

A pesar de las ventajas ambientales de los veh\'iculos el\'ectricos, uno de los retos m\'as importantes radica en la gesti\'on \'optima de la energ\'ia almacenada en sus bater\'ias. Un problema frecuente en este tipo de sistemas es la presencia de arm\'onicos, que son fluctuaciones no deseadas en la corriente el\'ectrica. Estos arm\'onicos afectan tanto el rendimiento del sistema como la vida \'util de las bater\'ias, lo que subraya la necesidad de una soluci\'on eficaz para mitigar este problema. Seg\'un Palafox (2009), los medios de transporte son responsables del 50\% de la contaminaci\'on mundial, lo que pone en evidencia la importancia de encontrar alternativas m\'as limpias y eficientes. Diversos autores han demostrado que los arm\'onicos no solo reducen la eficiencia, sino que tambi\'en generan p\'erdidas t\'ermicas considerables en los convertidores (Marcos-Pastor et al., 2015; Orellana Ugu\~na et al., 2022).
\setlength{\parskip}{1em}

Existen estudios recientes que han explorado la reducci\'on de arm\'onicos en sistemas de carga mediante diversas t\'ecnicas de filtrado. Sin embargo, la mayor\'ia de estos enfoques no han logrado optimizar de manera significativa la eficiencia de los filtros, especialmente en escenarios de carga r\'apida. El uso de filtros LCL y t\'ecnicas de control avanzado ha mostrado buenos resultados, pero requiere un dise\~no cuidadoso del sistema y de sus par\'ametros (Zhang et al., 2021; Fern\'andez et al., 2022). La funci\'on de transferencia, como herramienta matem\'atica clave en el an\'alisis de sistemas din\'amicos, ofrece una forma eficaz de modelar y optimizar el comportamiento de estos filtros, mejorando su desempe\~no en diversas condiciones. En este proyecto, se pretende utilizar la funci\'on de transferencia en los filtros el\'ectricos en cargadores de bater\'ias para veh\'iculos el\'ectricos, no solamente optimizando su rendimiento, sino tambi\'en en la estabilizaci\'on de la corriente, lo que ayuda a evitar sobrecargas, reducir los arm\'onicos y contribuir a un futuro m\'as sostenible en el transporte.
\setlength{\parskip}{1em}

Esto evidencia que es necesario complementar el dise\~no de filtros con estrategias que aborden integralmente los efectos de los arm\'onicos en sistemas de carga. En ese contexto, se ha documentado que ``los filtros de arm\'onicas son esenciales para mitigar las distorsiones en sistemas HVDC, mejorando as\'i la calidad de la energ\'ia transmitida'' (Rogers Acevedo, 2008, p. 15). Adem\'as, el uso de modelos con funci\'on de transferencia no solo potencia la capacidad predictiva del comportamiento din\'amico del sistema, sino que tambi\'en permite dise\~nar filtros adaptativos que se ajusten a distintas condiciones operativas. A esto se suma la importancia de implementar controles que compensen componentes de secuencia negativa y cero, garantizando as\'i una operaci\'on m\'as estable de las microrredes (Universidad de Chile, 2015, p. 22).

\subsection*{Pregunta de investigaci\'on}
\textit{“¿C\'omo podr\'ia el uso de la funci\'on de transferencia en un filtro electr\'onico mitigar los arm\'onicos en los sistemas el\'ectricos de carga de veh\'iculos el\'ectricos?”}

\section{Objetivos}
\label{sec:objetivos}

\subsection*{Objetivo general}
\textit{Optimizar el dise\~no y an\'alisis de filtros electr\'onicos reduciendo los arm\'onicos en sistemas de carga de bater\'ias para veh\'iculos el\'ectricos, con el fin de mejorar la eficiencia y la estabilidad del proceso de carga.}

\subsection*{Objetivos espec\'ificos}
\begin{itemize}
  \item Analizar c\'omo la funci\'on de transferencia se aplica al dise\~no de filtros electr\'onicos para mitigar los arm\'onicos en sistemas de carga de bater\'ias para veh\'iculos el\'ectricos.
  \item Revisar estudios sobre el impacto de los arm\'onicos en la eficiencia de la carga de bater\'ias de veh\'iculos el\'ectricos, bas\'andose en la relaci\'on entre el an\'alisis de circuitos el\'ectricos a base de las leyes de Kirchhoff.
  \item Demostrar que la aplicaci\'on de la funci\'on de transferencia en los filtros de cargadores de bater\'ias mediante la formulaci\'on de una ecuaci\'on diferencial, bas\'andose en los modelos te\'oricos presentados en investigaciones previas.
  \item Revisar el impacto de los convertidores de potencia en el dise\~no de cargadores de bater\'ias, espec\'ificamente en su funci\'on como controladores de factor de potencia (PFC) y reguladores de corriente.
\end{itemize}

\section{Justificaci\'on}
\label{sec:justificacion}

El crecimiento acelerado de la demanda de veh\'iculos el\'ectricos ha impulsado la necesidad de sistemas de carga m\'as eficientes y estables. Uno de los principales problemas en estos sistemas es la generaci\'on de arm\'onicos, que deterioran la calidad de la energ\'ia el\'ectrica y afectan negativamente tanto el funcionamiento del cargador como la durabilidad de la bater\'ia. En este contexto, el presente estudio busca aplicar la funci\'on de transferencia como una herramienta de an\'alisis y optimizaci\'on de filtros electr\'onicos, con el fin de mitigar dichos arm\'onicos y mejorar el rendimiento global del proceso de carga.

Adem\'as, esta investigaci\'on contribuye al avance tecnol\'ogico del sector del transporte sostenible, proporcionando una base te\'orica y pr\'actica para el dise\~no de sistemas de carga m\'as seguros y confiables. El enfoque basado en la funci\'on de transferencia permite un an\'alisis m\'as preciso del comportamiento din\'amico de los filtros, lo cual resulta esencial para la mejora continua de los cargadores de bater\'ias y, por ende, de la infraestructura vehicular el\'ectrica.
    
\newpage

\section{Marco te\'orico}
\label{sec:marco_teorico}

En el contexto de la movilidad el\'ectrica, el estudio de los sistemas el\'ectricos ha cobrado relevancia debido al creciente uso de veh\'iculos el\'ectricos, cuyos cargadores est\'an expuestos a perturbaciones como los arm\'onicos. Estas fluctuaciones en la corriente el\'ectrica afectan la calidad de la carga, reduciendo la eficiencia del sistema y acortando la vida \"util de componentes como las bater\'ias. Para mitigar estos efectos, se emplean herramientas matem\'aticas como la funci\'on de transferencia y la Transformada de Laplace, que permiten modelar el comportamiento del sistema y dise\~nar filtros electr\'onicos adecuados (Zhang, Wang, \& Li, 2021). Este marco te\'orico aborda los principios esenciales para dicho an\'alisis, incluyendo las leyes de Kirchhoff, fundamentales en el estudio de circuitos el\'ectricos.

\subsection{1. Conceptos}

\subsubsection{1.1 Funci\'on de Transferencia:}
La funci\'on de transferencia es una herramienta matem\'atica utilizada para describir la relaci\'on entre la entrada y la salida de un sistema din\'amico en el dominio de Laplace. Es especialmente \"util en la ingenier\'ia de control y en el an\'alisis de sistemas el\'ectricos, donde permite modelar la respuesta del sistema ante perturbaciones como los arm\'onicos. En el caso de los cargadores de bater\'ias de veh\'iculos el\'ectricos, la funci\'on de transferencia ayuda a modelar el comportamiento de los filtros electr\'onicos que mitigan los arm\'onicos presentes en el sistema de carga (Bola\~nos, s/f).

\subsubsection{1.2 Transformada de Laplace:}
La Transformada de Laplace es una herramienta esencial en el an\'alisis de sistemas din\'amicos, permitiendo convertir ecuaciones diferenciales en ecuaciones algebraicas. Este enfoque simplifica la resoluci\'on de problemas complejos en circuitos el\'ectricos y otros sistemas din\'amicos. En el contexto de los cargadores de bater\'ias, la transformada de Laplace es \"util para analizar c\'omo los filtros afectan la se\~nal de entrada, permitiendo el dise\~no de soluciones m\'as efectivas para mitigar los arm\'onicos (L\'azaro et al., 2016).

\subsubsection{1.3 Leyes de Kirchhoff:}
Las leyes de Kirchhoff son esenciales para la evaluaci\'on de circuitos el\'ectricos. As\'i mismo, la ley de corrientes de Kirchhoff (LCK) establece que la suma de las corrientes en un nodo es cero, mientras que la ley de voltajes de Kirchhoff dicta que la suma de los voltajes en una malla cerrada es cero. Estas leyes son clave para entender el flujo de energ\'ia en los sistemas de carga de bater\'ias y son esenciales en el dise\~no de filtros que ayudan a reducir los arm\'onicos (Arias, 2015).

\subsubsection{1.4 Arm\'onicos en Circuitos El\'ectricos:}
Los arm\'onicos son componentes de frecuencia m\'ultiplo de una se\~nal fundamental que se superponen a la se\~nal de corriente o voltaje en un sistema el\'ectrico. Estas frecuencias adicionales son generadas por la no linealidad de los dispositivos conectados al sistema, como convertidores de potencia o cargadores de bater\'ias, que generan distorsiones en la forma de onda de la corriente o el voltaje. Estos arm\'onicos afectan negativamente la calidad de la se\~nal el\'ectrica y pueden interferir con el funcionamiento adecuado de los componentes del sistema (Abundis, 2016).

\subsection{2. Campo de Aplicaci\'on Espec\'ifico}
El campo de aplicaci\'on espec\'ifico de este proyecto es el sistema el\'ectrico de carga de bater\'ias para veh\'iculos el\'ectricos. Este sistema es un ejemplo de un sistema din\'amico que se ve afectado por la presencia de arm\'onicos, los cuales son fluctuaciones en la corriente que pueden afectar negativamente la eficiencia de la carga y la vida \"util de las bater\'ias. El estudio se enfoca en c\'omo los filtros electr\'onicos, modelados a trav\'es de ecuaciones diferenciales y utilizando la funci\'on de transferencia, pueden mitigar estos arm\'onicos, mejorando as\'i el rendimiento del sistema de carga.

\subsubsection{2.1 M\'etodos de Resoluci\'on:}
\textbf{I. An\'alisis mediante la funci\'on de transferencia:}\\
Para resolver el problema de los arm\'onicos en los sistemas de carga, se utilizar\'a la funci\'on de transferencia para modelar el comportamiento de los filtros electr\'onicos. Esto permitir\'a predecir la respuesta del sistema ante perturbaciones y optimizar el dise\~no de los filtros.
\vspace{1em}

\textbf{II. Resoluci\'on mediante la Transformada de Laplace:}\\
Se aplicar\'a la Transformada de Laplace a las ecuaciones diferenciales que describen los circuitos el\'ectricos, transform\'andolas en ecuaciones algebraicas m\'as f\'aciles de manejar (Fern\'andez et al., 2022). Este m\'etodo es crucial para analizar c\'omo los filtros electr\'onicos pueden mitigar los arm\'onicos y mejorar la eficiencia del sistema de carga.
\vspace{1em}

\textbf{III. Simulaci\'on y an\'alisis de estabilidad:}\\
Se realizar\'an simulaciones del sistema de carga de bater\'ias, modelado mediante la funci\'on de transferencia y las ecuaciones diferenciales, para evaluar el impacto de los arm\'onicos y la efectividad de los filtros. Adem\'as, se analizar\'a la estabilidad del sistema utilizando herramientas matem\'aticas basadas en las leyes de Kirchhoff y otros m\'etodos de resoluci\'on anal\'itica.
\vspace{1em}

\textbf{IV. Aplicaciones complementarias:}\\
Se revisar\'an estudios previos sobre estrategias de compensaci\'on activa y control adaptativo que permitan un mejor rendimiento en condiciones variables. Se tomar\'an como referencia modelos similares en microredes y sistemas HVDC (Rogers Acevedo, 2008; Universidad de Chile, 2015).

\newpage

\section{Formulaci\'on del problema}
\label{sec:formulacion_problema}

En este proyecto, se estudiar\'a un convertidor boost, que es un tipo de convertidor DC-DC utilizado en cargadores de bater\'ias para veh\'iculos el\'ectricos. En este an\'alisis, se modelar\'an las p\'erdidas por conducci\'on de los componentes clave como la bobina ($L$), condensador ($C$), diodo directo ($D$) y mosfet activado ($Q$), adem\'as de incluir un modelo de la bater\'ia compuesto por una resistencia $R_{bat}$ en serie con una fuente de voltaje $V_{bat}$. El modelo presentado a continuaci\'on considera supuestos como la linealidad del sistema, y las limitaciones incluyen la no consideraci\'on de efectos no lineales complejos que podr\'ian ocurrir en condiciones extremas.

El modelo est\'a representado por las siguientes ecuaciones de peque\~na se\~nal:

\[
V_L(s) = V_G(s) - L_L(s) \cdot R_{eq} - V_D \cdot D' - V_0 \cdot D'
\]

\[
I_C(s) = \frac{V_0(s) - V_{bat}}{R_0} + I_L(s) \cdot D'
\]

Donde:

\begin{itemize}
  \item $V_L(s)$ \quad \textendash\quad Es la tensi\'on en la bobina.
  \item $V_G(s)$ \quad \textendash\quad Es la fuente de entrada.
  \item $I_L(s)$ \quad \textendash\quad Es la corriente de la bobina.
  \item $V_D$ \quad \textendash\quad Es el voltaje del diodo.
  \item $D'$ \quad \textendash\quad Es el ciclo de trabajo complementario del PWM.
  \item $V_0(s)$ \quad \textendash\quad Es la tensi\'on de salida.
  \item $R_{eq}$ \quad \textendash\quad Es la resistencia equivalente del circuito.
  \item $R_0$ \quad \textendash\quad Es la resistencia de carga.
  \item $I_L(s)$ \quad \textendash\quad Es la corriente de la bobina.
\end{itemize}

Este conjunto de ecuaciones describe el comportamiento del convertidor en condiciones de peque\~na se\~nal, permitiendo analizar la respuesta del sistema ante las fluctuaciones. El modelo de peque\~na se\~nal se utiliza para identificar los efectos de las p\'erdidas por conducci\'on en el circuito de carga y su influencia sobre la eficiencia del sistema de carga de bater\'ias.

El an\'alisis de los arm\'onicos y su filtrado en este sistema es fundamental para mejorar la eficiencia del cargador. Los arm\'onicos generados en el sistema pueden alterar la calidad de la energ\'ia suministrada a la bater\'ia y, por lo tanto, influir en su vida \'{u}til y rendimiento.

Un par\'ametro clave en este modelo es la resistencia de carga $R_0$ y la resistencia interna de la bater\'ia $R_{bat}$. Estos par\'ametros se estudiar\'an con mayor profundidad m\'as adelante, ya que son cruciales para entender c\'omo las p\'erdidas de energ\'ia afectan la eficiencia global del convertidor y el tiempo de vida \'{u}til de la bater\'ia. Tambi\'en se investigar\'a el efecto del ciclo de trabajo del PWM en la respuesta del sistema.

\section{Referencias bibliogr\'aficas}
\label{sec:referencias}

\begin{itemize}
  \item Abuinas, J. (2016). \textit{Electrical Harmonics in Power Systems}. IEEE Access.
  \item Arias, F. (2015). \textit{Fundamentos de Circuitos El\'ectricos}. Editorial Alfaomega.
  \item Bola\~nos, J. (s/f). \textit{Introducci\'on al An\'alisis de Sistemas Din\'amicos}. Universidad Nacional.
  \item L\'azaro, P., Fern\'andez, M., y Torres, J. (2016). \textit{Estudio del filtrado de arm\'onicos mediante la transformada de Laplace}. Revista de Ingenier\'ia El\'ectrica.
  \item Agencia Internacional de Energ\'ia. (2023). \textit{Global EV Outlook}. Recuperado de https://www.iea.org/reports/global-ev-outlook-2023
  \item Palafox, C. (2009). \textit{Impacto ambiental del transporte y energ\'ia}. Universidad de Guadalajara.
\end{itemize}


\end{document}
