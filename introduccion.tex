\section{Introducci\'on}
\label{sec:introduccion}

El transporte es un componente esencial de la sociedad moderna, ya que facilita el movimiento de personas y bienes, impulsa la econom\'ia y favorece la interacci\'on social. Con el tiempo, los veh\'iculos se han convertido en el principal medio para satisfacer estas necesidades. Actualmente, el sector atraviesa una transformaci\'on impulsada por el auge de los veh\'iculos el\'ectricos, que, al funcionar con energ\'ia el\'ectrica en lugar de combustibles f\'osiles, representan una alternativa m\'as sostenible y contribuyen a la disminuci\'on de la contaminaci\'on ambiental. Seg\'un la Agencia Internacional de Energ\'ia (2023), la adopci\'on de veh\'iculos el\'ectricos podr\'ia reducir las emisiones de carbono en un 30\% para el 2040, lo que destaca a\'un m\'as la importancia de esta transici\'on.

A pesar de las ventajas ambientales de los veh\'iculos el\'ectricos, uno de los retos m\'as importantes radica en la gesti\'on \'optima de la energ\'ia almacenada en sus bater\'ias. Un problema frecuente en este tipo de sistemas es la presencia de arm\'onicos, que son fluctuaciones no deseadas en la corriente el\'ectrica. Estos arm\'onicos afectan tanto el rendimiento del sistema como la vida \'util de las bater\'ias, lo que subraya la necesidad de una soluci\'on eficaz para mitigar este problema. Seg\'un Palafox (2009), los medios de transporte son responsables del 50\% de la contaminaci\'on mundial, lo que pone en evidencia la importancia de encontrar alternativas m\'as limpias y eficientes.

Existen estudios recientes que han explorado la reducci\'on de arm\'onicos en sistemas de carga mediante diversas t\'ecnicas de filtrado. Sin embargo, la mayor\'ia de estos enfoques no han logrado optimizar de manera significativa la eficiencia de los filtros, especialmente en escenarios de carga r\'apida. La funci\'on de transferencia, como herramienta matem\'atica clave en el an\'alisis de sistemas din\'amicos, ofrece una forma eficaz de modelar y optimizar el comportamiento de estos filtros, mejorando su desempe\~no en diversas condiciones. En este proyecto, se pretende utilizar la funci\'on de transferencia en los filtros el\'ectricos en cargadores de bater\'ias para veh\'iculos el\'ectricos, no solamente optimizando su rendimiento, sino tambi\'en en la estabilizaci\'on de la corriente, lo que ayuda a evitar sobrecargas, reducir los arm\'onicos y contribuir a un futuro m\'as sostenible en el transporte.

\subsection*{Pregunta de investigaci\'on}
\textit{“¿C\'omo podr\'ia el uso de la funci\'on de transferencia en un filtro electr\'onico mitigar los arm\'onicos en los sistemas el\'ectricos de carga de veh\'iculos el\'ectricos?”}
\newpage
